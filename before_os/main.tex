\documentclass{beamer}
\usepackage{hyperref, xcolor}

\hypersetup{
    colorlinks=true,
    linkcolor=blue,
    filecolor=magenta,      
    urlcolor=cyan
    }

\title{Before the OS class}

\author{Chih-Hsuan Yang(SCC)\\
{\small \href{mailto:zxc25077667@pm.me}{zxc25077667@pm.me}}
}

\begin{document}
\begin{frame}
    \maketitle
\end{frame}

\begin{frame}{Why this lession?}
    \centering
    Many students don't know what do the prof. Chiang saying.
\end{frame}

\begin{frame}{Install a Linux distro}
    \begin{itemize}
        \item Using {\color{orange} Ubuntu} is highly recommended. ({\small \href{https://www.ubuntu-tw.org/modules/tinyd0/}{Ubuntu latest} LTS version})
        \item The final {\color{orange} Ubuntu} 22.04{\tiny (Jammy Jellyfish)} release which is scheduled to {\color{red} April 21}, 2022
    \end{itemize}
    Other distro: \href{https://en.wikipedia.org/wiki/Linux_distribution}{Linux distributions}
\end{frame}

\begin{frame}{Is WSL acceptable?}
    Well, yes. Please see: \href{https://docs.microsoft.com/zh-tw/windows/wsl/kernel-release-notes}{WSL2-Linux-Kernel}.\\
    But, it would be more complicated.
\end{frame}

\begin{frame}{What is `shell'?}
    \centering
    Please see: \href{https://en.wikipedia.org/wiki/Shell_(computing)}{Wikipedia}
\end{frame}

\begin{frame}{Toward advanced C programming}

\end{frame}

\end{document}